\section{Conclusion}
\label{sec:conclusion}

In this paper we presented \dBoost/, a toolkit that leverages tuple expansion to detect outliers in both numerical and heterogeneous data sets. We demonstrated that well-known machine-learning strategies could be used to flag spurious numerical and to a lesser extent non-numerical data. We also demonstrated that simple correlation modeling is useful in inferring data dependencies and improving the accuracy of outlier detection procedures. We discussed histogram-based models, and showed that they provided a useful tool in analyzing mostly textual data.

We showed that our toolkit performs well on real-world problems, including identifying potentially wrong entries in a people directory and flagging erroneous values generated by faulty sensors. Our toolkit and its source code are available for public use under a permissive license, with the hope of allowing database users to formulate their own type-based rules and find discrepancies in their own data.
Once a large library of rules is developed, we anticipate new challenges such as how to select expansion rules for efficient analysis.

%We believe that the preliminary results presented in this paper are promising, especially in the area of identifying outliers in heterogeneous data.

